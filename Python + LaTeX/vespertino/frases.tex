\documentclass[12pt]{article}
\usepackage[margin=0.3in]{geometry}

\newcommand{\tipo}[2]{\textbf{#1 + #2}}
\newcommand{\parte}[1]{\underline{#1}}
\newcommand{\partes}[3]{{\sffamily \parte{#1} \textit{#2} \parte{#3}}}

\begin{document}

	Fazer como um teste de lógica, mas sem muito sentido. 
	
	Com várias casas, personagens e ações, sortearia um conjunto de dicas para dar associando duas informações (uma casa e um personagem, por exemplo).
	
	Eu gostaria que os testes tivessem as escritas de pessoas diferentes, e por isso quero convidar vocês. 
	Peço que preencham os espaços dos 9 modelos de frases abaixo com algo que se encaixe sintaticamente, como nos exemplos.
	
	Essas coisas não precisam ser iguais e coerentes. 
	Podem ser vários personagens diferentes no papel de sujeito, vários lugares diferentes, vários predicados diferentes. 
	Personagens não precisam ser pessoas ou animais, predicados não precisam ser domésticos e os lugares não precisam ser casas. 
	
	Essas coisas podem ser bem criativas, engraçadas ou sérias, a gosto.
	
%	Concordância e 

\vfill
\begin{enumerate}
\item \tipo{Lugar}{lugar}:	
	\subitem \partes{Uma casa amarela fica}{à esquerda}{da casa azul}	
	\subitem \partes{O sobrado vermelho está}{à direita}{dos correios}
	
	Flexione um verbo que concorde com o primeiro lugar e que possa ser usado com \textit{esquerda} ou \textit{direita}.
	Lembre-se que o segundo lugar deve usar o conectivo \textit{de} (ou variações como \textit{de um} e \textit{das}) para estar ao lado do outro.
	
\item \tipo{Lugar}{sujeito}:	
	\subitem \partes{Nos fundos do 201}{}{mora o cachorro grande}	
	\subitem \partes{Nas casas geminadas}{}{tem uma samambaia festeira}
	\subitem \partes{Em uma decrépita mansão}{}{o Fulano vive}
	
	Flexione um verbo que concorde com o sujeito e que indique localização.
	O verbo não precisa necessariamente aparecer antes do sujeito.
	
\item \tipo{Lugar}{predicado}:	
	\subitem \partes{No porão do Zé}{}{alguém caiu}	
	\subitem \partes{Em cima do muro de tijolos}{}{jogou-se cartas}
	\subitem \partes{Nas mesas da pracinha}{}{foi ouvido um barulho}
	
	Use sujeito indeterminado, expressões como \textit{alguma pessoa}, \textit{alguém} ou \textit{algo} ou voz passiva para que a ação tenha ênfase e o sujeito não seja detalhado.
	
\item \tipo{Sujeito}{lugar}:	
	\subitem \partes{O gato mora}{}{na árvore podada}	
	\subitem \partes{Tem uma pombinha mãe-solo}{}{em uma casa térrea alugada}
	\subitem \partes{O medo existencial vive}{}{na farmácia} 
	
	Flexione um verbo que concorde com o sujeito e que indique localização.
	O verbo não precisa necessariamente aparecer depois do sujeito.
	
\item \tipo{Sujeito}{sujeito}:	
	\subitem \partes{Uma pedra mora}{à esquerda}{do corvo}	
	\subitem \partes{A partitura para violoncelo vive}{à direita}{da minhoca que não gosta de maçã}	
	
	Flexione um verbo que concorde com o primeiro sujeito e que possa ser usado com \textit{esquerda} ou \textit{direita}.
	O verbo não precisa necessariamente aparecer depois do primeiro sujeito.
	Lembre-se que o segundo sujeito deve usar o conectivo \textit{de} (ou variações como \textit{de um} e \textit{das}) para estar ao lado do outro.
	
\item \tipo{Sujeito}{predicado}:	
	\subitem \partes{A barata}{}{diz que tem 7 saias de filó}	
	\subitem \partes{Ela}{}{tem 1 só}
	\subitem \partes{Beltrano}{}{da cama caiu}
	
	O verbo não precisa necessariamente aparecer no começo do predicado.
	
\item \tipo{Predicado}{lugar}:	
	\subitem \partes{Alguém quebrou um vaso sanitário}{}{no telhado da sobrinha da Maria}	
	\subitem \partes{Regaram as plantas}{}{no terreno baldio}
	\subitem \partes{As luzes foram acesas}{}{nas ruínas} 
	
	Use sujeito indeterminado, expressões como \textit{alguma pessoa}, \textit{alguém} ou \textit{algo} ou voz passiva para que a ação tenha ênfase e o sujeito não seja detalhado.	

\item \tipo{Predicado}{sujeito}:	
	\subitem \partes{Está alegre}{}{a galinha-d'angola}	
	\subitem \partes{Ouviram um brado retumbante}{}{as margens plácidas do Ipiranga}
	\subitem \partes{Bem alto gritou}{}{o Ciclano} 
	
	O verbo não precisa necessariamente aparecer no começo do predicado.
	
\item \tipo{Predicado}{predicado}:	
	\subitem \partes{Abasteceram o caminhão do João}{à esquerda}{de onde alguém pulava corda}	
	\subitem \partes{Um computador HP foi formatado}{à direita}{de onde mudou-se a senha do wi-fi}
	
	Conecte o segundo predicado de forma que possa ser usado com \textit{esquerda} ou \textit{direita}.
	Os verbos não necessariamente precisam aparecer nos começos dos predicados.
	
	
	% bueiro
\end{enumerate}
\end{document}
